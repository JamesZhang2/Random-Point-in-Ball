\usepackage{amsmath,amssymb}
%\usepackage{mathabx}
\usepackage[margin=1in]{geometry}
\usepackage{titlesec}
\usepackage{amsthm}
\usepackage{tikz-cd}
\usepackage[]{mathrsfs}
\usepackage{mathtools}
\usepackage{todonotes}
\usepackage{calc}
\usepackage{blindtext}

\usepackage{graphicx}
\graphicspath{ {./images/} }

\usepackage{hyperref}
\hypersetup{
  colorlinks=true,
  linkcolor=cyan,
  filecolor=red,
  urlcolor=blue,
}

%%%%%%%%%%%%%%%%%%%%%%%%%%%%%%%%%%%%%%%%%%%%%%%%%%%%%%%%%%%%%%%%%%
% Problem Counter

\newcounter{problem}[section]
\newenvironment{problem}[1][]{\refstepcounter{problem}\par\medskip
  \noindent \textbf{Problem~\theproblem. #1} \rmfamily}{\medskip}

%%%%%%%%%%%%%%%%%%%%%%%%%%%%%%%%%%%%%%%%%%%%%%%%%%%%%%%%%%%%%%%%%%
% Theorem Styles

\theoremstyle{plain}
\newtheorem{theorem}{Theorem}[problem]
\newtheorem{lemma}[theorem]{Lemma}
\newtheorem{corollary}[theorem]{Corollary}
\newtheorem*{theorem*}{Theorem}
\newtheorem*{lemma*}{Lemma}

\theoremstyle{definition}
\newtheorem{proposition}[theorem]{Proposition}
\newtheorem{definition}[theorem]{Definition}
\newtheorem{conjecture}{Conjecture}[section]
\newtheorem{question}{Question}[section]
\newtheorem*{question*}{Question}
\newtheorem{example}[theorem]{Example}

\theoremstyle{remark}
\newtheorem*{claim}{Claim}
\newtheorem*{remark}{Remark}
\newtheorem*{note}{Note}
\newtheorem*{exercise}{Exercise}
\newtheorem*{notation*}{Nt}

\renewcommand*{\proofname}{Proof}

%%%%%%%%%%%%%%%%%%%%%%%%%%%%%%%%%%%%%%%%%%%%%%%%%%%%%%%%%%%%%%%%% 
% mathbb

\newcommand{\N}{\mathbb{N}} % natural numbers
\newcommand{\Z}{\mathbb{Z}} % integers
\newcommand{\R}{\mathbb{R}} % real numbers
\newcommand{\Q}{\mathbb{Q}} % rational numbers
\newcommand{\C}{\mathbb{C}} % complex numbers
\newcommand{\E}{\mathbb{E}} % expected value

%%%%%%%%%%%%%%%%%%%%%%%%%%%%%%%%%%%%%%%%%%%%%%%%%%%%%%%%%%%%%%%%%
% Math operators

\DeclareMathOperator{\im}{Im} % image
\DeclareMathOperator{\tr}{tr} % trace
\DeclareMathOperator{\id}{id} % identity map (lowercase)
\DeclareMathOperator{\Id}{Id} % identity map/matrix
\DeclareMathOperator{\rank}{rank} % rank
\DeclareMathOperator{\sgn}{sgn}
\DeclareMathOperator{\sign}{sign}
\DeclareMathOperator{\sech}{sech}
\DeclareMathOperator{\csch}{csch}

%%%%%%%%%%%%%%%%%%%%%%%%%%%%%%%%%%%%%%%%%%%%%%%%%%%%%%%%%%%%%%%%%%
% blu-bird & James

\newcommand{\eps}{\varepsilon}
\newcommand{\vphi}{\varphi}
\newcommand{\ind}[1]{\mathbf{1}_{#1}}
\newcommand{\vol}{\text{vol}}
\newcommand{\dpav}[1]{\mathcal{D}_{#1}}
\newcommand{\dvol}[2]{|d^{#1}\mathbf{#2}|}

\newcommand{\bfx}{\mathbf{x}}
\newcommand{\bfy}{\mathbf{y}}
\newcommand{\bfz}{\mathbf{z}}
\newcommand{\bfu}{\mathbf{u}}

\newcommand{\xypoint}[2]{\ensuremath{\begin{pmatrix} #1 \\ #2 \end{pmatrix}}}
\newcommand{\xyzpoint}[3]{\ensuremath{\begin{pmatrix} #1 \\ #2 \\ #3 \end{pmatrix}}}
\newcommand{\xyzwpoint}[4]{\ensuremath{\begin{pmatrix} #1 \\ #2 \\ #3 \\ #4 \end{pmatrix}}}
\newcommand{\xymatrix}[2]{\ensuremath{\begin{bmatrix} #1 \\ #2 \end{bmatrix}}}
\newcommand{\xyzmatrix}[3]{\ensuremath{\begin{bmatrix} #1 \\ #2 \\ #3 \end{bmatrix}}}
\newcommand{\xyzwmatrix}[4]{\ensuremath{\begin{bmatrix} #1 \\ #2 \\ #3 \\ #4 \end{bmatrix}}}
\newcommand{\xyrowmat}[2]{\ensuremath{\begin{bmatrix} #1 & #2 \end{bmatrix}}}
\newcommand{\xyzrowmat}[3]{\ensuremath{\begin{bmatrix} #1 & #2 & #3 \end{bmatrix}}}

\newcommand{\pzpx}{\frac{\partial z}{\partial x}}
\newcommand{\pzpy}{\frac{\partial z}{\partial y}}
\newcommand{\ppzpxx}{\frac{\partial^2 z}{\partial x^2}}
\newcommand{\ppzpxy}{\frac{\partial^2 z}{\partial x \partial y}}
\newcommand{\ppzpyy}{\frac{\partial^2 z}{\partial y^2}}

\newcommand{\ignore}[1]{}  % from Charles
\newcommand{\parens}[1]{\ensuremath{\left( #1 \right)}}
\newcommand{\bracks}[1]{\ensuremath{\left[ #1 \right]}}
\newcommand{\braces}[1]{\ensuremath{\left\{ #1 \right\}}}
\newcommand{\set}[1]{\braces{#1}}
\newcommand{\powset}[1]{\mathcal{P}\parens{#1}}

\newcommand{\floor}[1]{\left\lfloor #1 \right\rfloor}
\newcommand{\ceil}[1]{\left\lceil #1 \right\rceil}
\newcommand{\verts}[1]{\left\lvert #1 \right\rvert} % | #1 |
\newcommand{\Verts}[1]{\left\lVert #1 \right\rVert} % || #1 ||
\newcommand{\abs}[1]{\verts{#1}}
\newcommand{\size}[1]{\verts{#1}}
\newcommand{\norm}[1]{\Verts{#1}}

\newcommand{\extd}{\mathbf{d}}

\renewcommand{\implies}{\Rightarrow}
\renewcommand{\iff}{\Leftrightarrow}

\newcommand{\collab}[1]{I discussed this problem set with #1.}
